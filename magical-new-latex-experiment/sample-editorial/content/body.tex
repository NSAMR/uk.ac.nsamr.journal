Research holds a significant role in the careers of medical professionals, something that is highlighted by ``Clinical Research and Scholarship'' and ``Patient Safety and Quality Improvement'' being recognised as key sections of the \cite{GeneralMedicalCouncil2018} guidelines: ``Outcomes for graduates 2018''.

A portion of trainee doctors choose to pursue research out of interest, but for many the simple reality is that demonstrating involvement in research is an integral part of the job application process. This starts early, with points being awarded for publishing in a PubMed-indexed journal on application to foundation training \citep{UKFoundationProgramme2018}, and continues through a doctor's career progression to core then specialty training \citep{HealthEducationEngland2018,HealthEducationEngland2018a}.

In response, organisations such as the \citeauthor{NSAMR2018} and INSPIRE by \citeauthor{TheAcademyofMedicalSciences2018}, have been founded to promote research amongst medical students, which have created collaborative projects that aim to provide students with the experience of research, such as \citeauthor{STARSurg2018}. Further, universities promote participation in research through intercalated degrees, compulsory audit, elective periods, and student selected modules, some of which may be research specific \citep{Lundin2016}.

However, while these do promote research as a whole, they do not address the publication of research specifically: training in publication and writing is uncommon in student run organisations; collaborative projects output impressive results but do `not develop the student's understanding of academia'; and only a few students actually participate in the publication process \citep{Ashcroft2017}. Overall, medical students receive minimal training in the ins and outs of publication as part of their degrees.

The result of this is that many medical students do not write up their audit or research project, and the vast majority (86\,\%) do not publish anything while at university \citep{Griffin2011}. This represents a wasted learning opportunity. Students miss out on developing an understanding of the publication process, particularly how to prepare a manuscript and respond to reviewers' comments or undertake a peer review. Additionally, there are a number of other barriers to publication that students face: few journals publish case reports and audits, which are common opportunities available to medical students; competition with established researchers; and page or open access fees mean that a large number of journals are inaccessible to students as authors. The culmination of these factors is that while many students understand the importance of research they are underprepared in how to publish any research they undertake.

It is no wonder therefore, that \cite{Glasziou2018} say there is a ``scandal'' of research wastage, which occurs due to research not being published, design flaws in studies, and material which has been published but is incomplete or unusable. In fact, they estimate around 50\,\% of research is not published, resulting in wasted effort and money, unnecessary duplication of research, and most importantly --- no clinical benefit \citep{Glasziou2016}. Thus, there is a clear need to improve the training that medical students receive in publication, so in 2016, the National Student Association of Medical Research founded a journal to address this.

At first it might seem as though another journal is not needed. There are plenty of journals already in existence, and some authors even suggest that there are too many \citep{Oosterhaven2015}. There are a number of existing journals aimed at medical students, and searching the first five pages of Google --- accounting for the majority of the hyperlinks individuals use \citep{Chaffey2018} --- for ``UK student medical journal'' reveals that there are eight academic journals for medical students targeted solely at the UK audience \citep{BritishMedicalJournal2018,CambridgeMedicalJournal2018,INSPIRE2018,JournaloftheRoyalMedicalSociety2018,ManchesterMedicalJournal2018,TheBritishStudentDoctorJournal2018}. But the question remains: why should a medical student submit to a student journal when they could submit their project to a larger, more well-established journal? Indeed, none of the journals currently available are indexed in PubMed, and realistically these journals are unlikely to gain one in the near future, due to the stringent requirements for registration. The application involves ``at least a two-year history of quality scholarly publishing in the life sciences, [and] a minimum of 25 peer reviewed articles (e.g., original research or review articles, clinical case reports)'', for review and ``approximately 50 articles'' for technical evaluation \citep{PMCUSNationalLibraryofMedicineNationalInstitutesofHealth2018}. Few student journals have been extant for longer than two years, or have 50 articles in their archives. Those that do have been around for over a decade, and so have likely either applied and failed, or it is not an urgent priority for them.

Student journals should not make PubMed indexing their primary objective. Current student journals should not try to emulate existing journals, as in doing so they do not add anything new. There is no need for another journal of this description. Instead student journals should evaluate alternative ways to distinguish themselves from pre-existing well-established journals.

None of the current student journals offer openly accessible guidance with formal training in peer review or the publication process. There are also pitfalls surrounding sustainability, as one student journal, ``\citeauthor{ScottishUniversitiesMedicalJournal2014}'' has not published for four years. As such, sustainability must be built into the ethos of a student journal, something that is particularly important when a journal is formed by volunteers independent financial support and the support of a university or senior academics, such as ours.
Therefore, this journal will fulfil an unmet need as it will provide: a platform for medical students to publish their research, to prevent wasted or duplicated effort; training in submission, peer review, and editing, to help prevent the submission and acceptance of flawed study design; use of standards for the publication of research, to reduce the potential for incomplete or unusable material; diamond open access (no fee to submit and no fee to view) to increase accessibility; only publishing work that was completed as a student to reduce competition from established researchers; and to maintain sustainability through partnership with the National Student Association of Medical Research and use of open access resources where possible.