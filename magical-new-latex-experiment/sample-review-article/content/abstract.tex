Development of a revolutionary targeted therapy --- imatinib --- dramatically improved survival for the thousands of patients suffering from the formerly fatal chronic myeloid leukaemia. Yet, this revolutionary therapy fails in a third of all cases, prompting us to consider future treatment goals.

By delving into the key translocation and consequent chimeric oncogene underlying this myeloid malignancy, this review introduces the means by which imatinib succeeds and fails, subsequently considering the outcomes and cost-efficacy of a second-generation drug: dasatinib. In light of the recent large-scale DASISION trial, this review aims to explore the effectiveness of dasatinib in treating cases where imatinib has failed, and finally examines the evidence behind its very recent approval as a first-line therapy in England. Ultimately, the review looks ahead to a new drug --- ponatinib --- and questions our future treatment goals; how do we refine our targeted therapies when they fail?
	
	